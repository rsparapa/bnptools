
The instrumental variable (IV) approach has long been a cornerstone of
causal inference from both the theoretical and applied perspectives.
For example, the distribution-free method of two-stage least squares
(TSLS) goes back to \cite{Thei53} and it is based on earlier IV work
that goes back decades further such as \cite{Wrig28}.  The focus on
distribution-free methods is paramount since the reliance on
parametric assumptions has been roundly criticized \citep{LaLo86}.
Therefore, there has been a movement towards nonparametric methods
that do not rely on precarious restrictive assumptions such as
functional forms and/or convenient choices of distributions
\citep{AngrImbe95}.  Conversely, unrestricted nonparametric approaches
may have theoretical challenges such as the lack of causal
identification \citep{Pear09}.  While in practical performance,
distribution-free methods such as TSLS have come under attack as more
biased and less powerful than Ordinary Least Squares (OLS) with
standard errors generated by either bootstrapping or jack-knifing
\citep{Youn19}.

We take a Bayesian approach to IV as many others have before us
\citep{ImbeRubi97,RossAlle05,CHMR08,ROSSI14}.  For example,
\cite{RossAlle05} take a Bayesian parametric approach that we will
refer to as {\it linear-normal} or {\it lin-nor} (for linear IV with
normal errors) based upon the linear structural equations of TSLS.
\cite{CHMR08} expand on this previous work via a semi-parametric
method that relaxes the parametric error distribution with Dirichlet
Process Mixtures (DPM) \citep{EW95} while retaining the linear model
structural equations of TSLS: we will refer to this method as {\it
linear-DPM} or {\it lin-DPM}.  For a comprehensive exposition of the
IV framework from the Bayesian perspective, along with Bayesian
nonparametric priors, see \cite{ROSSI14}.

Herein, we propose a new nonparametric method based on Bayesian
Additive Regression Trees (BART) \citep{ChipGeor10} and DPM
capable of handling structural equations that may be non-linear and/or
may have non-normal errors.  We will refer to our new method as {\it
IVBART} which we describe in Section~\ref{flex-mod}.  In
Section~\ref{simulated}, we explore our new method with simulated data
sets and compare with lin-nor and lin-DPM.  Section~\ref{card} is
where we delve into a real data set that
%where we delve into the well-known \cite{Card95} data set that
demonstrates our new method to estimate the monetary returns of
post-secondary education (as have others \citep{Card93,CHMR08}).  In
Sections~\ref{gibbs} and \ref{details},
%post-secondary education.  In Sections~\ref{gibbs} and \ref{details},
we provide the details for implementing our new method via Markov
chain Monte Carlo (MCMC) sampling of the posterior.
Section~\ref{sec:conclusion} concludes the article with a brief
discussion of the merits of our new method and some potential future
directions for extensions.  In the Appendix, we provide a brief
introduction to the {\bf ivbart R} package that implements our new
method and a proof of the causal identification of our new method.

