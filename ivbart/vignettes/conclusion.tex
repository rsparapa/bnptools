
The linear instrumental variables model has long been fundamental in causal analysis.
It simply and elegantly captures the fundamental intuition
that an instrumental variable $z$, may provide
a source of variation in a treatment $T$, comparable to that of an experiment
in which
variation is induced by an investigator who
controls the value of $T$. 

However the assumption of linearity is rarely one that we can  comfortably impose.
In practice, this usually leads to a search for a set of transformations of the instruments
$z$ and the additional variables $x$ which are then used in the linear setting.
Even with modern methods for finding transformations this process is tedious and depends on
choices for the set of transformations considered.

Our use of Bayesian Additive Regression trees (BART) allows us to capture a wide
range of possible functions with no user input and still do a full Bayesian analysis including nonparametric
modeling of the error terms.

For our nonparametric error term analysis we have followed \cite{CHMR08} closely given its success.
This as led to a prior-sensitivity approach in which we vary the prior beliefs about the nonlinear functions $f$ and $h$.
Also, our goal here is inferential in that we seek to learn $\beta$ while in BART, the prior development has been
more focused on the goal of out of sample prediction.
The BART models for these two functions allow for a relatively simple scheme for varying our prior beliefs.
We hope that the top-left plot of Figure~\ref{fig:alldraws} and the analysis of the Card data in Figure~\ref{fig:card-data-2w-sens-1}
will suggest to practictioners that IVBART provides a relatively simple alternative to the difficult challenges presented
by the general sensitivity of inference for the treatment effect $\beta$ to the model specification.

In future work we will consider the use of more informative data based priors for the error distribution 
as in \citep{ChipGeor10} and \citep{DPMBART}.
In addition, future work will seek to relax the additive linear assumption for the treatment effect.
While are current analysis is very flexible and allows for simple interpretation of the causal effect through
the parameter $\beta$, we wish to consider the possibility of hetergeneous treatment effects.
We note that our current model is already very flexible and powerful and extentions to a still more flexible model
will entail careful prior choices as in \citep{BCF}.

We note that the simlulation study presented in this paper provides further support for the
efficacy of the linear approaches provided by the \texttt{R} package
\texttt{bayesm} \citep{RPbayesm} in the functions
\texttt{rivGibbs} for the linear model with correlated normal errors and \texttt{rivDP} for the linear model
with nonparametrically modeled errors.


%Our new approach offers a dramatic improvement in model flexibility to
%applied researchers with very little extra effort expended.

% Future work notes
% More settings for beta such as 0 and -1
% A table of the bias for simulation settings
% Informative priors for beta
% Relaxing linearity assumption for beta that is identifiable
% More functionality of the ivbart package like multi-threading,
% variable selection, etc.
